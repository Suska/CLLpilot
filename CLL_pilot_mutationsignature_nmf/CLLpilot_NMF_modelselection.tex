\documentclass[a4paper,11pt]{article}
\usepackage[T1]{fontenc}
\usepackage[utf8]{inputenc}
\usepackage{lmodern}
\usepackage{textcomp}
\usepackage{amssymb}
\usepackage[margin=1.5cm]{geometry}
\usepackage{lscape}

\title{CLL pilot -  Non-negative matrix factorization (NMF)}
\author{Dr. Susanne Weller}
\date{\today}

\usepackage{Sweave}
\begin{document}
\input{CLLpilot_NMF_modelselection-concordance}

\maketitle

\section{Data preparation}
\begin{Schunk}
\begin{Sinput}
> #Working directory
> setwd("/home/andreas/suska/work/02_CLLpilot/CLLpilot/trunk/CLL_pilot_mutationsignature_nmf")
> #input file
> nmfdata <- read.table("CLLpilot_mutationsignature.tsv", header=TRUE, sep="\t")
> names(nmfdata)[1] <- "patientID" 
> row.names(nmfdata)<- nmfdata$patientID
> nmfdata$patientID <- NULL
> nmfdata <-t(nmfdata)
\end{Sinput}
\end{Schunk}
\section{NMF-optimization}
NMF model was optimized by repeating the model 1000 times for ks between 2 and 20. The optimal k was selected where the cophenetic correlation coefficient fell sharply.(DOne on workstation)
The final model was the best selected from these 1000 models.
\section(final NMF model evaluation)
\begin{Schunk}
\begin{Sinput}
> library(NMF)
> #k=4
> set.seed(1712)
> mutsignaturenmf <- nmf(nmfdata, 4, seed="random", method="brunet", nrun=1000, .options="p4")
> #The fitted model can be retrieved via method fit, which returns an object of class NMF:
> targetmatrix <- fit(mutsignaturenmf)
> targetmatrixSignatures <- as.data.frame(targetmatrix@W)
> #This converts the counts into percentages
> targetmatrixSignatures[1:4] <- as.data.frame(lapply(targetmatrixSignatures[1:4], function(x)(x/sum(x))*100))
> targetmatrixPatients <- as.data.frame(targetmatrix@H)
\end{Sinput}
\end{Schunk}

\subsection{Mutational Signatures for Each patient}
